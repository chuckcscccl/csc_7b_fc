\documentstyle[11pt,proof]{article}
\bibstyle{plain}
\parindent=0.0cm
\parskip=0.4cm
\setlength{\textwidth}{15cm}
\setlength{\oddsidemargin}{+0.4cm}
\setlength{\evensidemargin}{0.4cm}
\setlength{\topmargin}{+0.0cm}
\setlength{\headheight}{0cm}
\setlength{\textheight}{22cm}
\setlength{\headsep}{1cm}
\renewcommand{\baselinestretch}{1.0}
\def\nvec#1,#2{#1_{1}, \ldots, #1_{#2}}

\pagestyle{plain}

%\marginparwidth 0pt \oddsidemargin 0pt
%\evensidemargin 0pt \marginparsep 0pt
%\topmargin 0pt

%\textwidth 6.5in \textheight 8.5in

%\setlength{\parskip}{6pt}
%\pagestyle{plain}
%\newcommand\Pscr {{\cal P}}
\newcommand\rrr  {\rightarrow}
\newcommand\lrr  {\longrightarrow}
\newcommand\lrrs {\stackrel{\ast}{\longrightarrow}}
\newcommand\lrrb {\stackrel{\ast}{\longleftrightarrow}}

\begin{document}

\begin{center}
{\large\bf Class Notes on Lambda Calculus}\\[5pt]
  {\normalsize\bf Chuck Liang}\\[5pt]
  {\normalsize\bf Hofstra University Computer Science}
\end{center}

\normalsize

\bigskip
\section{Background and Introduction}

The Lambda Calculus is an abstract, mathematical basis for studying properties
of programming languages just as integral calculus is used to study physics and
other sciences.  It was invented by Alonzo Church, who along with Alan Turing is
considered one of the founders of computer science.
%Although Church
%originally created the $\lambda$-calculus to study the foundations of mathematics,
%its impact on computer science has been much greater.
We will study the basics of the lambda calculus for two
reasons.  First, most modern programming languages have direct support
for ``lambda expressions'', and they are increasingly being used in
advanced programming.  Secondly, and {\em far more importantly,\/} the
lambda calculus is a theoretical foundation for most of computer science:
all computer algorithms can be explained by the lambda
calculus (this is part of what is called the {\em ``Church-Turing
  Thesis.''\/}) It explains the behaviors of most programs and
languages even when they don't use lambda expressions explicitly.

There are two versions of the lambda calculus: typed and untyped.  They
roughly correspond to statically typed (compiler-based) and
dynamically typed (interpreter-based) languages.  This document will
first focus on the untyped lambda calculus.  Typed lambda calculi will be
briefly covered at the end.

\medskip

The syntax of the lambda calculus will look strange at first.
Essentially it is trying to answer the question: {\em how do we apply
  a program to input?\/} We can study this problem in the form of {\em
  how to apply a function to an argument.\/} In many languages we
write this as $f(t)$, where f is the function and t is the argument.
In lambda calculus we write this as $(f~ t)$.  Essentially, applying a
program (function) to an input (argument) requires ``substitution,''
which is to just {\em replace each occurrence of the formal argument
  with the actual argument.\/} The lambda calculus is the smallest and
the purest programming language. It is purely mathematical in that it
is not concerned about how computation is to be implemented
physically, only how symbols are to be transformed.

Each language has a syntax for specifying functions.  In lambda
calculus functions are represented using the $\lambda$ binder.  So
\verb+'def f(x): return x*x'+ in Python, for example, will be $f =
\lambda x.(*~x~x)$.  The $\lambda x.$ ``binder'' defines the (formal)
argument to the function.  The $\lambda$-term itself is a value,
sometimes called a {\em nameless function,\/} and we are just using
the symbol $f$ to refer to it.  When we apply the function to an
(actual) argument, like in $(f~4)$, we replace each $x$ in the body of
the function with $4$, which results in $4*4$, or $(*~4~4)$: this
expression must be evaluated further using the built-in function for
$*$.  The evaluation continues until we reach a {\em ``normal
  form,''\/} such as $16$, which can't be evaluated any further.
Sounds simple enough, but there are some complications.

Actually, in Python you can also just write \verb+'f=lambda x:x*x'+,
but even in languages that do not support lambda
syntax directly, the lambda calculus is still significant.
Consider the following C programs:
\begin{verbatim}
int f(int x) 
{
   int y = 1;
   { 
     { // starts inner scope
        int x = 2;
        y=y+x;
     }
   }
   return y+x; // which x is this referring to?
}
\end{verbatim}
What should happen if we applied $f(4)$?  We can't just replace all $x$'s with $4$:
we need to recognize that there are two $x$'s in the body, and only the outer one
corresponds to the formal argument.  $f(4)$ should return 7, but only if we
respect the {\em scope\/} of variables.  Consider another example:
\begin{verbatim}
int y = 1; // variable declared external to f (such as a global)
int f(int x)
{
   int y = 8;
   return x+y;
}
\end{verbatim}
What should happen if we applied $f(y+1)$?  You should know that it will
return 10, but only because you understand that there are two $y$'s and
which $y$ is being referred to in $x+y$. If we just blindly replaced $x$ with
$y+1$, the function would return 17.

Finally, consider the complete C program:
\begin{verbatim}
#include<stdio.h>
int x = 1;
int f(int y) { return x+y; }
int main()
{
   int x = 2;
   printf("%d\n", f(0) ); //What will this print? What SHOULD it print?
   return 0;
}
\end{verbatim}
Which $x$ is the $x$ in the body of $f$ referring to? Why?

To handle these situations correctly, we need to enhance our strategy to
replace formal arguments with an actual arguments.  First, we
need to distinguish between local variables (called ``bound variables''
in the lambda calculus) and non-local variables (called ``free variables'').


\section{Formal Definitions}

An expression (term) in the lambda calculus is defined inductively as follows:
\begin{enumerate}
\item a variable, $x$, $y$, $z$ etc.
\item $(A ~B)$, where $A$ and $B$ are both lambda terms.  This is called an
  ``application''.
\item $\lambda x.A$, where $x$ is a variable and $A$ is a lambda term.  This is
  called an ``abstraction.''
\item  A known constant such as $3$.  Constants are not strictly necessary but
  we will include them in order to show more interesting examples.  Constants
  can also be functions: for example, $*$ is used to represent the multiplication
  function in most languages.
\end{enumerate}
For example, $\lambda x.(y~x)$ is a lambda term because $x$ and $y$ are variables,
so $(y~x)$ is a term and therefore $\lambda x.(y~x)$ is also a term.

According to this definition, $(3~3)$ is a valid lambda term, and is read as
``three applied to itself.''  That's fine.  The lambda calculus is a purely
symbolic, that is to say purely syntactic, system.  Do not try to attach any
{\em meaning\/} to the symbols and rules just yet. 

{\em Syntactic Conventions:}  Applications in lambda terms associates to the
left, and applications bind tighter than abstractions.  This means that
$(a~b~c)$ is equivalent to $((a~b)~c)$ and {\bf not} to $(a~(b~c))$. The parentheses
are dropped when not needed.  The term $\lambda x.x ~y$ should be understood to be
equivalent to $\lambda x.(x~y)$ and {\em not\/} to $(\lambda x.x)~y$, because
application binds tighter than abstraction: in other words, the scope of the
$\lambda x$ binder extends to the right until it's delimited
by enclosing $()$s.  Thus in \ $(\lambda x.a)~(b~c)$ \ both pairs of
parentheses are needed, but in \ $\lambda x.((a~ b)~c)$ \ both are
superfluous.

Here's another way to understand the syntactic conventions.  Write
applications $A~B$ as $app(A,B)$ and write abstractions $\lambda x.A$
as $abs(x,A)$.  Then {\em application associates to the left\/} means
that $A~B~C$ represents $app(app(A,B),C)$ and not
$app(A,app(B,C))$. {\em Application binds tighter than abstraction\/}
means that $\lambda x.A~B$ is $abs(x,app(A,B))$ and not
$app(abs(x,A),B)$.  

Given an abstraction $\lambda x.M$ (where $M$ is an arbitrary
term), $M$ is called the ``body'' (or {\em scope\/}) of the $\lambda$-abstraction and $\lambda x$ is
the ``binder.''  We can think of such an abstraction as a function that takes x
then evaluates and returns M.  

\subsection*{Free and Bound Variables}

Inside a given term, variables can be local or non-local.  Local variables
are those that occur inside the scope of a $\lambda$-binder, and free variables
are those that do not.  For example, in $\lambda x.(y~x~x)$, $x$ is bound and
and $y$ is free.  Whether a variable is free or bound depends on the exact
term in question: in the entire term $\lambda x.(x~y)$, $x$ is bound, but
in the body $(x~y)$, $x$ is free.  The variable $x$ that forms a part of the
binder $\lambda x$ can also be consider bound, but it's not part of the
body of the term (it declares that a variable is bound in the body).
Consider the more interesting example:
$$ \lambda x. y~ (\lambda x.x)~ x $$
Observe that by the syntactic conventions, this term should be
read as $\lambda x. ((y~ (\lambda x.x))~ x) $.
In the entire term, all
occurrences of $x$ are bound and $y$ is free. Inside the body of the
term $y~ (\lambda x.x)~ x $, however, the right-most occurrence of $x$ is free
but the one inside the parentheses is bound.  The right-most $x$ 
refers to the outer $\lambda x$ binder.

\subsection*{Substitution and Beta-Reduction}

Now we can more formally define how to apply a function in the form of a $\lambda$-abstraction to
an argument.  First, we define {\em substitution:\/}.  Given a term
$M$, let $M[N/x]$ represents the term obtained by replacing {\em all free
  occurrences of $x$ in $M$ with $N$\/}.  Furthermore, we require the following
condition: {\em the free variables of $N$ cannot appear bound in $M$.\/}

For example, the substitution $((\lambda x.x)~ x) [N/x] $ yields the
term $(\lambda x.x)~ N$: only the {\em free\/} occurrence of $x$ in
the term is replaced with $N$.  The bound instance of $x$ in the term
represents a local variable, and should not be replaced: in other
words, $\lambda x.x$ is self-contained (the proverbial ``black box'').

Now we define the most critical rule of Lambda Calculus, 
called {\bf ``beta reduction''}:
$$ (\lambda x.M)~N  ~~\Rightarrow_\beta~~  M[N/x]$$
That is, to apply a $\lambda$-abstraction (function) to an argument, substitute
$N$ for $x$ in M under the following two conditions:
\begin{enumerate}
\item Only the free occurrences of $x$ in $M$ are to be replaced ($x$ is bound
  in the entire term $(\lambda x.M)$ but it may be free in the body $M$).
\item The free variables contained in $N$ may not be bound in $M$.
\end{enumerate}
A term in the form $(\lambda x.M)~N$ is called a {\em beta-redex:\/} a place
where beta-reduction can be applied.
Consider some examples:
\begin{itemize}
  \item $(\lambda x.x+x)~ y$ beta reduces ($\Rightarrow$) to $y+y$.  Here, $M$ is $x+x$
  and $N$ is $y$ and we replace both free occurrence of $x$ in $M$ with $y$.

\item  $(\lambda x.y)~z  ~~\Rightarrow~~ y$ .  Here, $M$ is $y$ and $N$ is $z$.  There
  are no free $x$'s in the body, so there's nothing to replace: $y[z/x]$ is still $y$.
  Such lambda terms correspond to functions that do not use their arguments.
  The $\lambda x$ binder here is {\em vacuous.\/}

\item   $(\lambda x\lambda y.y~ x)~ u ~(\lambda v.v) ~~\Rightarrow ~~ u$.  It is tempting
  to think of a term that begins with $\lambda x\lambda y\ldots$ as a ``function that takes two arguments,'' but it
  is actually a function that takes one argument and returns another function,
  which takes the other argument.  Such functions are called ``Curried'' functions
  (referring to the logician Haskell Curry).  In this example, we first
  replaced $x$ with $u$, resulting in the intermediate term
  $(\lambda y.y~u)~(\lambda v.v)$.  For this beta-redex, the argument ($N$) is
  itself a lambda-term, and we replace $y$ with $\lambda v.v$.  This results
  in the term $(\lambda v.v)~u$.  One final reduction reduces this to $u$.
  In other words, the reduction steps are:
  $$(\lambda x\lambda y.y~ x)~ u~ (\lambda v.v) ~\Rightarrow ~
  (\lambda y.y~u)~(\lambda v.v) ~\Rightarrow ~ (\lambda v.v)~u \Rightarrow u$$
  Note: although the parentheses around $\lambda v.v$ are not required
  in $(\lambda y.y~u)~(\lambda v.v)$, it's better to place them there because
  after substitution its scope should be clear.

\item  The next term corresponds to the second C program given in the
  introduction:
  $$(\lambda y\lambda x.(\lambda y.x+y)~8)~4$$  Please
read the parentheses carefully: there's an inner beta-redex, which is
$(\lambda y.x+y)~8$ where $M$ is $x+y$ and $N$ is 8, and an outer beta-redex,
where $M$ is $\lambda x.(\lambda y.x+y)~8$ and $N$ is $4$.  It does not
matter which redex we reduce first (this is called the {\em confluence property.\/}). Let us reduce the outer redex first:
this means substitute all free occurrences of $y$ in the body with $4$, but
there are actually no free occurrences of the outer $y$ in the body: the
$y$ inside the body is bound locally.  The reduction sequence is therefore:
$$(\lambda y\lambda x.(\lambda y.x+y)~8)~4 ~\Rightarrow~
\lambda x.(\lambda y.x+y)~8 ~\Rightarrow~  \lambda x.x+8
$$
In the last step, we reduced the beta-redex inside the outer $\lambda x$, which
is $(\lambda y.x+y)~8$.
\end{itemize}
In each case above we reduced the term until there are no beta-redexes anywhere
in the term, resulting in a {\em beta normal form.\/}

{\em Are all terms reducible to normal forms?\/} {\bf NO.} The standard example
is
$$
(\lambda x.x~x) ~(\lambda x.x~x) ~\Rightarrow~
(\lambda x.x~x) ~(\lambda x.x~x)
$$
%$$
%(\lambda x.x~x) ~(\lambda x.x~x)
%$$
Yes, this is {\em a term that applies its argument to itself and is itself applied to itself, and
such a term reduces to itself.\/} Got it?  Don't even try to understand it
intuitively: just apply the rules of beta reduction.
Both free occurrences of $x$ in the body $x~x$ are replaced by the argument,
which is $(\lambda x.x~x)$ (which does not contain any free variables so the
restrictions on substitution do not apply).
The term reduces back to itself and therefore will never be in normal form.
The existence of such terms is important if we are to have repetition
(loops, recursion) in our programs.
%The above term is also used in the proof of the undecidability
%of the Halting Problem (in later section).


{\flushleft\bf Bad Examples of Beta Reduction:}
\begin{itemize}
\item $(\lambda y.\lambda x.x~y)~ v ~\Rightarrow~ (\lambda x.x)~ v
  ~\Rightarrow~ v$.  This is {\bf wrong} because we misused the
  syntactic conventions of the lambda calculus.  The first reduction
  results in $\lambda x.x~v$, which is not the same as $(\lambda x.x)~v$ because
  application binds tighter than abstraction: it is equivalent to
  $\lambda x.(x~v)$, which is already in normal form.

\item $ (\lambda x.y~ x)~ (\lambda u.u) ~z  ~\Rightarrow (\lambda x.y~ x)~z
  ~\Rightarrow y~z$.  This is wrong because application should associate
  to the left: the correct reduction sequence is
$ (\lambda x.y~ x)~ (\lambda u.u) ~z  ~\Rightarrow y~ (\lambda u.u)~z$,
which is already in normal form.

\item  $(\lambda y.\lambda y.y~y)~ v ~\Rightarrow~ (\lambda y.v~v)$.  We
  didn't make the same mistakes as above but we made a worse
  one. We can only replace free occurrences of $y$ inside the
  body of the outer $\lambda$, which is $\lambda y.y~y$.  There are no
  {\em free\/} occurrences of $y$ in this term!  The correct normal form
  is $\lambda y.y~y$.

\item $(\lambda y.\lambda x.x~y)~ x ~\Rightarrow~ (\lambda x.x~x)$.
  In the lambda calculus, this is as bad a mistake as you could possibly
  make.  You violated the second restriction to beta reduction: {\em the
    free variables of $N$ cannot appear bound in $M$.\/}  $N$ in this case is
    the free variable $x$, but inside the body of the $\lambda$-term, $x$
    is bound (a local variable).  From a programmer's perspective, this restriction
    should be intuitive: the local (bound) variable $x$ should not be confused with
    externally defined (free) variables by the same name.  The wrong reduction
    does what's called ``free variable capture.''  Free variables must stay
    free after reduction.
\end{itemize}
So how do we reduce terms like in the last example?  We just
{\em rename\/} the bound variable to separate it from the free
variable.

\subsection*{Alpha Conversion}

Surely \verb+int f(int y) {return y;}+ and 
\verb+int f(int x) {return x;}+  \ are the same program.  Not only do they behave
the same, but they are syntactically the same: the only difference is in the
choice of the local (bound) variable used.  In lambda calculus, terms such as
$\lambda x.x$ and $\lambda y.y$ are called {\em alpha-equivalent.\/}  Each
term can be {\em alpha-converted\/} to the other by replacing the names of
bound (local) variables consistently.  Formally, a variable $y$ is called
{\em fresh\/} with respect to a term $M$ if $y$ does not appear anywhere in $M$:
it's a new variable\footnote{technically, it can appear in some places but to
  simplify things, let's assume that it's a completely new variable not found anywhere
  in $M$.}.  Alpha conversion is defined by:
$$\lambda x.M  ~~\Rightarrow_\alpha~~ \lambda y.M[y/x] $$
where $y$ is {\em fresh\/} with respect to $M$.

It is important to remember that alpha-equivalence is a purely
syntactic equivalence and does not assume anything about the {\em
  meaning\/} of symbols.  For example, $\lambda x\lambda y.x+y$ and
$\lambda y.\lambda x.y+x$ are alpha-equivalent (one can be converted
to the other by first replacing one of the variables with an
intermediate fresh variable $z$).  However, $\lambda x\lambda y.x+y$
and $\lambda x.\lambda y.y+x$ are {\bf not} alpha-equivalent.  First,
you cannot assume that the symbol '$+$' represents numerical
addition, which is commutative.  In many languages $+$ is used for
string concatenation which is not commutative ("ab" $\not=$ "ba").  Secondly, even
if $+$ is known to be commutative the two terms are still not
alpha-equivalent because they are structurally different.  Alpha-equivalence
is a purely syntactic property.

We need alpha-conversion to allow beta reduction to proceed correctly when
the names of free and bound variables clash.  In the example
$(\lambda y.\lambda x.x~y)~ x$, we can apply alpha conversion to the lambda
term and rewrite it as $(\lambda y.\lambda z.z~y)~ x$, which then reduces
to $\lambda z.z~x$: the free variable $x$ is not illegally captured by the
local binder.  This example roughly corresponds to the situation found in
the first C program of the introduction.

We can only change the names of {\bf bound\/} variables,
not free variables.  It would be illegal to change the outer (rightmost) $x$ to
something else.  Free variables correspond to external variables which may
be in use elsewhere, not just {\em locally.\/}


\section{Basic Combinators}

A lambda term without any free variables (completely self-contained) is
called a {\em combinator.}  There are three famous ones:
\begin{enumerate}
\item $I ~=~ \lambda x.x$
\item $K ~=~ \lambda x\lambda y.x$
\item $S ~=~ \lambda x\lambda y\lambda z.x~z~(y~ z)$  
\end{enumerate}
It is know that all other combinators can be generated by combining these
terms in some fashion.  Let's look at how they behave. First, it should
be clear, if you understood the above, that $((\lambda x.x)~ A)$ beta-reduces
to $A$, regardless of the what kind of term $A$ is.  So we can write down
an {\bf axiom:} $IA = A$.
Now consider $K$: this terms takes two arguments (actually one after the other),
and ignores the second argument.  Given any two terms $A$ and $B$ where $y$
does not appear free in $A$ (alpha-convert $K$ otherwise), $(K~A~B)$ reduces
to $A$:
$$(K~A~B) = (\lambda x\lambda y.x)~A~B ~\Rightarrow~ (\lambda y.A)~B
~\Rightarrow~ A$$
In the last step, because $y$ is not free in $A$, $A$ is not affected by the
substitution $A[B/y]$.  Since K can always be alpha converted away from the
free variables of $A$, we can write down another {\bf axiom:} $KAB = A$.

Now consider $(K~ I~ A~ B)$ where $I$ is (alpha-equivalent to) $\lambda u.u$
and $A$ and $B$ are arbitrary.  Recall that by syntactic convention, this is
the term $((K~ I)~A)~B$.  First we reduce the innermost redex $(K~I)$:
$$KI = (\lambda x\lambda y.x)I ~\Rightarrow~ (\lambda y.I) = \lambda y.\lambda v.v$$ This terms takes two arguments and this time it ignores the first argument.
We should see that
$KIAB = (\lambda y\lambda v.v)AB ~\Rightarrow~ (\lambda v.v) B = B$.
So we've another {\bf axiom:} $KIAB=B$.
We can alpha convert $KI$ to $\lambda x\lambda y.y$.

It may appear that terms (such as $K$ and $KI$), which ignore some of their
arguments are pointless. But in fact, they correspond to something you're already
very familiar as programmers.  Given two choices $A$ and $B$, $K$ {\em chooses\/}
$A$ and $KI$ chooses B.  We are now beginning to see how lambda terms correspond
to constructs found in every programming language, in this case Boolean
expressions and {\bf if-else.\/}
But let's first look at an example involving $S$:
$$
SKI = (\lambda x\lambda y\lambda z.x~z~(y~ z))~K~I ~\Rightarrow
(\lambda y\lambda z.K~z~(y~z))~ I ~\Rightarrow
\lambda z.K~z~(I~z)  ~\Rightarrow \lambda z.z$$
In the last step, we applied the axiom $KAB=A$ where $A$ is $z$ and
$B$ is $(I~z)$.  Alternatively, we can first reduce $(I~z)$ to $z$ with the
axiom $IA=A$, but the result will be the same.

\section{More Powerful Combinators}

When Church invented the lambda calculus, he wasn't just thinking about computing.
There was a hypothesis (called ``Hilbert's Program'') that all mathematical
problems can be algorithmically solved.  Eventually, this hypothesis was proven
to be false by Kurt Godel and his {\em incompleteness theorems.\/}  But
many (though not all) mathematical properties can be described by the
lambda calculus.  The most basic computational operations are adding and
multiplying numbers ...

\subsection{Church Numerals}

Church formulated a representation of natural numbers as lambda terms:
zero is $\lambda f\lambda x.x$, one is $\lambda f\lambda x.f~x$, two is
is $\lambda f\lambda x. f~(f~x)$, three is $\lambda f\lambda x. f~(f~(f~x))$, and
so on.  Why did he write numbers this way?  Because he wanted to study
the {\em algorithmic\/} properties of numbers.  The ``meaning'' of a number
is, to him, defined by what we do with them, i.e., arithmetics. 
With this representation of numbers, basic arithmetic operations can be defined as lambda terms:
\begin{itemize}
\item $PLUS = \lambda m\lambda n\lambda f\lambda x.m~f~(n~f~x)$
\item $TIMES = \lambda m\lambda n\lambda f\lambda x.m~(n~f)~x$
\end{itemize}
Other operations are also possible (the one for subtraction is pretty hard).
For example, we can show that $1+2=3$ using beta-reduction:
$$(PLUS~1~2) = \lambda f\lambda x. 1~ f~(2 ~ f~x) ~\Rightarrow
\lambda f\lambda x. 1~f~(f~(f~x)) ~\Rightarrow
\lambda f\lambda x. f~(f~(f~x)) = 3$$
I first substituted $1$ and $2$ for $m$ and $n$ respectively, then
reduced $(2 ~ f~x)$ to
$f~(f~x)$.  Plus works by substituting one number into the end of the
other number.  Times works by replacing each $f$ with more $f's$.  As
an exercise, you can verify that $(TIMES~2~3)=6$.

We won't go further into Church numerals except to say that all basic
arithmetic operations can be represented as pure lambda calculus.  Even though
in actual languages we don't use lambda terms to represent numbers, lambda
calculus provides a theoretical unity to different kinds of computation.

\subsection{Booleans and If-Else}

A programming language needs more than just numbers and arithmetic.  Church's
philosophical approach to thinking about numbers extends to truth and
falsehood.  What is the significance of true and false from a purely
computational standpoint? It depends on what we can do with true and false.
As programmers, you certainly know what to do with these booleans: we use them
to make decisions.  From this algorithmic standpoint, true and false
just represent two opposing choices:
\begin{itemize}
\item $TRUE = K = \lambda x\lambda y.x$
\item $FALSE = KI = \lambda x\lambda y.y$
\end{itemize}
Then the familiar {\em if-else\/} construct can be defined by:
\begin{itemize}
  \item $IFELSE = \lambda c\lambda a\lambda b.(c~a~b)$
\end{itemize}
IF-ELSE takes three (Curried) arguments: a boolean expression $c$ that
should reduce to $TRUE$ or $FALSE$, and two choices $a$ and $b$.  Since
$Kab=a$ and $KIab=b$, true {\em chooses\/} a and false {\em chooses b}.
Thus $(IFELSE~ TRUE ~1 ~2)$ will beta-reduce to $1$ and
$(IFELSE~ FALSE ~1 ~2)$ will beta-reduce to $2$.

The lambda calculus should start to look less strange to you now ...

What about the boolean operations and, or and not? All of these can be
defined in terms of if-else.  For example:
\begin{itemize}
\item $NOT = \lambda a.(IFELSE~ a ~FALSE ~TRUE)$ 
\item $AND = \lambda a\lambda b.(IFELSE~a~b~FALSE)$
\end{itemize}
That is, $NOT$ is defined as a function that takes $a$ as an argument
and has body \verb+if (a) return false; else return true+.
$AND$ is a function that takes $a$ and then $b$ with body
\verb+if (a) return b; else return false+.
For example, $(AND~ TRUE ~ FALSE)$ beta-reduces to
(IFELSE ~TRUE~ FALSE~FALSE), which reduces to FALSE.
If you're not
convinced, do truth tables to see that these pure lambda terms do indeed have
the expected behaviors of the boolean operations.

As an exercise, define $OR$.


\subsection{Data Structures and Encapsulation}

How do we {\em construct\/} a single structure from multiple
components, and how do we {\em destruct\/} (break down) such a
structure into its components?  This sounds like a challenge to lambda
calculus, but all we need to do is to represent a pair of values,
which we normally write as $(a,b)$.  Then, using nested pairs, we can
construct data structures of any complexity.  For example, a linked list
is just a sequence of nested pairs ending in some designated
value representing the empty list: $(a,(b,(c,(d,null))))$.
For historical reasons the three operations we define are called {\em cons,\/}
{\em car\/} and {\em cdr:}
\begin{itemize}
\item $CONS = \lambda a\lambda b\lambda c.(c~a~b)$
\item $CAR = \lambda p.(p~TRUE)$
\item $CDR = \lambda p.(p~FALSE)$
\end{itemize}
Given two terms $A$ and $B$, $(CONS~A~B)$ returns (beta reduces to) a pair represented
by $\lambda c.(c~ A~ B)$.  {\em ``This is not a pair, it's a function!''\/}
says you.  Yes, but this function has all the behaviors we want from a pair.
The
argument $c$ is expected to be a boolean value and recall that $(c~a~b)$
is equivalent to $(IFELSE~c~a~b)$: true selects $a$ and false selects $b$.
Given such a ``pair'' represented by $P=\lambda c.(c~ A~ B)$, \ 
$(CAR~P)$ reduces to $(P~TRUE)$, which then reduces to $(TRUE~A~B)$. Thus
$(CAR~P)$ returns $A$. $(CDR~P)$ will similarly return $B$.  $CONS$ constructs
the structure and $CAR$, $CDR$ destruct it.

To represent linked lists, we also need to choose a term for the empty list
(typically called {\em null\/} or {\em nil:}) as well as a way to determine
if a pair is empty or not.
\begin{itemize}
\item $NIL = \lambda x.\lambda y.y$  \ (same as $FALSE$ and zero).
\item $ISNIL = \lambda p.p ~(\lambda x\lambda y.\lambda z.FALSE)~ TRUE$
\end{itemize}
$(ISNIL ~NIL)$ reduces to $NIL~(\lambda x\lambda y.\lambda z.FALSE)~ TRUE$,
which reduces to TRUE because NIL ignores the first argument
and returns the second ($KIAB=B$). But given a non-empty pair
$P=\lambda c.(c~A~B)$, $(ISNIL~P)$  reduces to:
$$P~(\lambda x\lambda y.\lambda z.FALSE)~TRUE ~\Rightarrow
(\lambda x\lambda y.\lambda z.FALSE)~A~B~TRUE ~\Rightarrow FALSE.$$
Now we can define a linked list as in
$$M = (CONS ~2 ~(CONS~3~(CONS~5~(CONS~ 7~ NIL))))$$
and write expressions such as $(CAR~(CDR~ M))$, which extracts the
second value from the list (verify that it reduces to 3). We can also write
$IFELSE ~(ISNIL~ M) ~A~ B$.
\ Add ``syntactic sugar'' to make it look like your favorite language.

Once we have pairs we can build other structures of arbitrary complexity.
For example, binary tree nodes can be represented by
$(CONS~X~(CONS ~LEFT~RIGHT))$.

\medskip

Representing data structures using functions is not so strange once you
realize that this is exactly what we want when constructing an {\em ``abstract
  data type.''}  For example, we often construct a {\em class\/} that contains
private values and public access functions.  The public functions form the
{\em interface\/} to the data type: the values that make up the data structure
are thus {\em encapsulated.}


\subsection{Repetition (Recursion)}

We need another component to create a
basic programming language: recursion.  The usual loop constructs of ordinary
languages can be seen as special cases of recursion. The idea is to create
a {\em fixed point operator\/} $fix$ with the behavior that $fix~M$
reduces to $M~(fix~M)$.  This will allow us to repeat arbitrarily many 
$M$s, or form loops that repeatedly evaluates $M$.  Combined with $IFELSE$, we
can terminate the loop when some condition is reached.
\begin{itemize}
  \item $FIX = \lambda m.(\lambda x.m~(x~x))~(\lambda x.m~(x~x))$
\end{itemize}
This term is similar to the term we saw earlier that reduces to itself
and has no normal form.  We verify that:
$$
(FIX~M) \Rightarrow (\lambda x.M~(x~x))~(\lambda x.M~(x~x)) ~\Rightarrow~
M~((\lambda x.M~(x~x))~(\lambda x.M~(x~x))) = M~(FIX~M)
$$
If it ever seemed strange to you that you can define a function that
refers to itself, the $FIX$ operator doesn't actually do that: it's applied
to a function that {\em names\/} the recursive function using a
$\lambda$-binder.
The following recursive program computes the sum of all numbers in a
linked list:
$$SUM = FIX~ (\lambda f.\lambda n.IFELSE~~ (ISNIL~ n)~~ ZERO ~~(PLUS~ (CAR~ n)~
(f~ (CDR~n)))$$
The recursive function is called by the $\lambda$-bound variable $f$.
To see how recursion works in more detail, here $M$ is
$\lambda f.\lambda n.IFELSE~ (ISNIL~ n)~ ZERO ~(PLUS~ (CAR~ n)~
(f~ (CDR~n))$ and $SUM = FIX~M \Rightarrow M~(FIX ~M)$.
For example, 
$$SUM~ (CONS~2~NIL) \Rightarrow M~(FIX~M)~ (CONS~2~NIL) \Rightarrow$$
$$IFELSE~(ISNIL~ (CONS~2~NIL))~ZERO ~ (PLUS~ 2~(FIX~M ~NIL)) \Rightarrow$$
$$ (PLUS~ 2~(FIX~M ~NIL)) \Rightarrow (PLUS~ 2~(M~(FIX~M) ~NIL)) \Rightarrow
(PLUS~2~0)\Rightarrow 2$$
The recursion stoped and returned $0$ when $(ISNIL~NIL)$ reduced to $TRUE$.  
We note that all elements of this program are definable in pure lambda calculus.


\section{Towards Programming Languages}

The previous section showed how common programming language constructs can
be represented by lambda terms.  Towards building a complete programming
language, we must also address several additional issues.

\subsection{Order of Evaluation}

As examples such as $SKI$ show, there are often multiple beta-redexes inside a
term and there are different strategies for selecting which one to reduce first.
Most popular programming languages use {\em call-by value\/} evaluation,
which means that the actual arguments to a function are fully evaluated
before they're passed to the function.  The alternative evaluation method
is {\em call-by-name\/}, which always reduces the outermost beta redex
first.  Consider $S(KI)K$.  If we reduced $(KI)$ first, it would be
call-by-value, and if we first substituted $(KI)$ into $S$, it would
be call-by-name.  Either strategy will give us the same normal form
$\lambda z.\lambda y.z$ (which is alpha-equivalent to $K$ again - verify
this normal form as an exercise).

{\flushleft\bf The Church-Rosser Theorem:} {\em If $A\Rightarrow_\beta B$ and
$A\Rightarrow_\beta C$, then there is a term $D$ such that
$C\Rightarrow_\beta D$ and $B\Rightarrow_\beta D$.}

In particular, if a term can be reduced to a normal form, then it must
be unique.  This property is also called {\em confluence\/} and it
says that all reduction strategies are valid, though not equally
efficient: some strategies may require more steps than others.
Consider $\lambda x.(x~x~x~x~x~x~x)~ (SKI)$.  Using call-by-value,
we only have to reduce $SKI$ to a normal form once, but using call-by-name,
we will have to reduce it six times.  This example suggests that call-by-value
is more efficient.  But that's not always true: consider
$$(\lambda y.z) ((\lambda x.x~x)~(\lambda x.x~x))$$  Using call-by-name,
the term reduces in one step to $z$, because $y$ is a vacuous binder.  How
many steps will call-by-value take?  Find the answer yourself, but don't take
forever :).

{\flushleft\bf Theorem: } If a term can be reduced to a normal form, the normal
form can be reached using the call-by-name strategy.

Redundant computations in call-by-name can also be avoided with more
sophisticated implementations:
if we used directed acyclic graphs {\em (DAGs)\/} instead of {\em trees\/} to
represent terms, then redundancies such as in $\lambda x.(x~x~x~x~x~x~x)$
can be avoided.  The main difficulty in implementing call-by-name is that
we must {\em delay\/} computation until it's actually needed.  This means we
can't just pass a value to a function, but must pass unexecuted {\em code\/}
to a function.  Furthermore, the code must carry the environment under which
it's defined in order to be executed correctly: it's called a {\em closure.\/}

Most conventional languages use call-by-value.  A notable exception is Haskell.
Some newer languages (Scala) now allow functions to be defined to use either
call-by-value or call-by-name.  Most conventional languages also do not
reduce terms that are hidden inside $\lambda$ abstractions: that is, in
$(\lambda x. (\lambda y.y)~x) A$, the beta-redex inside the $\lambda x$ term,
$(\lambda y.y)~x$ is not evaluated until the outer redex is reduced first.
In other words, we don't evaluate a function until it's passed all of its
arguments.  Also, certain built-in constructs must use call-by-name.  If we think
of $ifelse$ as a function of three arguments then under call-by-value
$(ifelse ~c~a~b)$ will evaluate both $a$ and $b$ regardless of the truth value
of $a$, and it is clearly not how we want if-else to behave.  Thus, from
a practical perspective, the $IFELSE$ combinator we defined earlier can only
be used with call-by-name.  Thus in most conventional languages, constructs
such as if-else (and short-circuited booleans) are
implemented as special constructs and not just as built-in functions.

To simulate the effect of call-by-name in the call-by-value 
setting of conventional languages, we have to encapsulate $b$ and $c$,
the terms that cannot be evaluated eagerly, inside vacuous 
$\lambda$-abstractions:
\begin{itemize}
\item $ifelse = \lambda c\lambda a\lambda b.(c~ a ~ b) ~I$
\end{itemize}
Instead of calling $(IFELSE ~ C~A~B)$, we call
$(ifelse~ C ~ (\lambda x.A) ~ (\lambda x.B))$. Here, $x$ is assumed to not
appear free in $A$ or $B$ (alpha-convert otherwise).
The dummy (vacuous) $\lambda x$
delays the evaluation of $A$ and $B$.  They are not evaluated until
$(c~a~b)$ is applied to $I$.

The FIX combinator requires a similar treatment.  The FIX combinator is
also called the $Y$ combinator:
\begin{itemize}
\item {\em Call-by-value FIX combinator:\/}
  $Y = \lambda m.(\lambda x.m~(\lambda y.x~x~y))~(\lambda x.m~(\lambda y.x~x~y))$
\end{itemize}

We will need these versions of the combinators in our implementation of
pure lambda calculus in real programming languages, such as Python and Perl.

\subsection{Static Scoping}

A critical subject we now address in lambda calculus concerns the third C program in the
introduction.  This program shows the difference between static (or lexical)
scoping and dynamic scoping. C, like most languages, is {\em statically\/}
scoped.  The third C program prints $1$ because the free variable $x$ in the
body of the function $f$ refers to the static context in which $f$ was {\em defined.\/}  With dynamic scoping, $f$ would refer to the nearest declared $x$ in
its {\em runtime\/} environment, which would be the $x$ declared in main.
Why is C (and most languages) statically scoped?  There are theoretical and
practical ways to explain this.  Theoretically, the lambda calculus shows
that it should definitely be statically scoped.

When we reduce $(\lambda x.M)~N$, $x$ is ``bound'' to $N$ and
then evaluated in $M$.  We can define a {\em let\/} construct that binds
variables to values in this way:
\begin{itemize}
\item 
  $(let~ x=A~ in~ B) ~=~ (\lambda x.B)~A$
\end{itemize}
We can now rewrite the third C program as follows:
$$
let~x=1 ~in~ (let~ f=(\lambda y.x+y) ~in~ (let ~x=2~ in~
(f~0)))
$$
This translates into the $\lambda$-term
$$
(\lambda x. (\lambda f.(\lambda x.f~0)~ 2)~ (\lambda y.x+y))~ 1
$$
Clearly, the $x$ inside $\lambda y.x+y$ refers to the outermost $\lambda x$.
The inner $\lambda x$ is a vacuous abstraction: this $x$ is not used anywhere.
This term beta-reduces to $1$  (verify this yourself).
The scoping rules of lambda calculus are those of static scoping.


\subsection{The Undecidability of the Halting Problem}

To complete this introduction to untyped lambda calculus, we prove in
pure lambda calculus the most important theoretical result in computer
science: the Halting Problem.  Programs can take other programs as
input and attempt to analyze their behavior: this is called {\em
  static analysis.}  Examples of such programs include compilers and
malware detectors.  The Halting Problem is: does there exists a
program that, given the source code of any program {\em as well as its
  input,\/} will return true if that program terminates (halts) on
that input, and false if that program does not terminate.  The problem
is {\em undecidable\/} because there cannot be such a program: the
very existence of such a program will lead to a logical contradiction.

The Halting problem can be formulated in terms of Turing Machines as well
as the (untyped) lambda calculus and several other equivalent formal systems.
The {\em Church-Turing Thesis\/} is that all algorithms can be formulated as
Turing machines, or as pure lambda terms.  Therefore, the following proof is
valid for all languages that claim to be ``Turing Complete:''

The proof is by contradiction ({\em reductio ad absurdum\/}): we make an assumption and show that the assumption
leads to an absurdity or paradox.
\begin{enumerate}
\item  Assume that there is a pure lambda term $HALT$ such that, given any
  lambda terms $P$ and $A$, $(HALT ~P~A)$ reduces to $TRUE$ if $(P~A)$ is
  reducible to a beta normal form, and that $(HALT ~P~A)$ reduces to $FALSE$ if
  $(P~A)$ cannot be reduced to a normal form.
\item Let $INF = (\lambda x.x~x)~(\lambda x.x~x)$. \  Recall that this term has no
  normal form: it does not halt.
\item Let $Q = \lambda p. (IFELSE~(HALT~p~p) ~~INF~~ I)$. \ \
  That is, if $p$ halts on itself as input, go into an infinite loop with $INF$, otherwise, return a normal form with $I$. 

\item Consider the question {\em DOES $(Q ~Q)$ HALT?.\/}  That is, is
  $(Q~Q)$ reducible to a normal form?  Based on the assumption that
  $(HALT~ Q~Q)$ returns $TRUE$ if $(Q~Q)$ halts and $FALSE$ otherwise,
  we see from the definition of $Q$ that
  \begin{quote}
    {\em if $(Q~Q)$ halts then $(Q~Q)$ does not halt, and if $(Q~Q)$ does not
      halt then it halts.\/}
    \end{quote}
\item We've reached a contradiction in that $(Q~Q)$ halts and does not halt.  
  The terms for $IFELSE$, $TRUE$, $FALSE$, $I$ and $INF$ are all pure lambda terms.
  The only other term we used was $HALT$ and its assumed behavior.  Therefore,
  the assumption that $HALT$ exists must be false.
\end{enumerate}

The Halting Problem defines the limits of computation based on the
Church-Turing model.  Many other undecidable problems are proved by
reducing them to the halting problem.  In practical terms, the
undecidibility of this problem means that we cannot always
determine how a program will behave {\em even if we know its input\/}
without actually running the program. But running the program could
result in an infinite loop.  This means we can't always experimentally
determine if there are errors in our programs.  Obviously,
these results are relevant to the design and use of programming
languages.

\section{Typed Lambda Calculus}

There are some important items about the lambda calculus that I did
not cover, including {\em ``eta-reduction:''\/} this rule just states
that $A = \lambda x.A~x$.  For example, the term for Church
mulitplication, $\lambda m\lambda n\lambda f\lambda x.m~(n~f)~x$, can
also be written equivalently as ($\eta$-reduces to) $\lambda m\lambda
n\lambda f.m~(n~f)$.  Most importantly, we did not cover {\em typed\/}
lambda calculus. A typed program is a logically consistent program: a
value cannot have conflicting types.  There is a close relationship
between type theory and mathematical logic.  The simplest typed lambda
calculus was also invented by Church and is known as ...

\subsection{The Simply Typed Lambda Calculus}


Types can
also be abstracted over (made generic).  The type of a function is written
$a\rightarrow b$ where $a$ is the type of the argument ($\lambda$-bound variable) and $b$ is the type of the body.  $I=\lambda x.x$, for example, has
generic type $a\rightarrow a$ and $K=\lambda x\lambda y.x$ has generic
type $a\rightarrow b \rightarrow a$, with the $\rightarrow$ operator
on types associating to the right.  Variables are associated with types
that they're assumed to have, using notation such as $x:int$.  From
a set of assumptions we derive a type on the right of the $\vdash$ symbol
using the following rules:
$$
\infer{x:a,\ldots \vdash x:a}{}
\qquad
\infer{\ldots \vdash \lambda x.M: a\rightarrow b}
      {x:a,\ldots \vdash M:b}
      \qquad
      \infer{\ldots \vdash (M~N): b}
      {\ldots \vdash M: a\rightarrow b &\quad \ldots \vdash N:a}
      $$
You should be able to derive the types for $I$ and $K$ using the
first two rules.
The last rule correspond to the classical syllogism
{\em modus ponens\/} and establishes the correspondence between types
and logic.  That is, if $\rightarrow$ can be read as {\em implies\/} then
the type of every combinator must be a propositional {\em tautology\/}
because the rules of typing are isomorphic to the rules of logic (this is
called the {\em Curry-Howard Isomorphism\/}).
The following is a sample type derivation using the rules:
$$
\infer{\vdash \lambda x\lambda y.y~x : a \rightarrow (a\rightarrow b) \rightarrow b}
      {\infer{x:a \vdash \lambda y:y~x : (a\rightarrow b) \rightarrow b}
        {\infer{x:a, y:a\rightarrow b\vdash (y~x) : b}
          {\infer{y:a\rightarrow b\vdash y:a\rightarrow b}{} &\quad
            \infer{x:a \vdash x:a}{}}}}
      $$
      You can verify that the derived type is a tautology by constructing a truth
      table.
      
An important consequence of introducing types is that beta-reduction
(and alpha/eta conversion) must preserve the types of terms:
{\flushleft\bf Theorem: Type Soundness.} {\em If $\vdash s:A$ is a valid type derivation
  for lambda term $s$ and $s\Rightarrow_\beta t$, then $\vdash t:A$ is also a
  valid type derivation.}

  This core result (also referred to as {\em subject reduction\/}) is what we
  mean by a computation being {\em type safe:\/} 
  3+1 is an expression of type int, and 4, the normal form, is also an int.
  So type checking can be done before, during, or after beta-reduction. If
  it's done before reduction, it's called {\em static typing.\/}
  
Another consequence of enforcing typing rules is that certain terms,
such as $\lambda x.(x~x)$, is not typable (does not have a type).  To
encode a programming language that allows for recursion and
repetition, the fixed pointer operator, with property $FIX~M = M~(FIX
~M)$, must be imported as an extra constant.  If we are to give $FIX$
a type, it would have to be $(a\rightarrow a)\rightarrow a$.  This is
a {\em not a tautology,\/} so no pure lambda term can have this type.


\subsection{Products, Disjunctions and Recursive Types}
\subsection{Quantification over Types}

\bigskip

\appendix

    \medskip   
    {\Large {\bf APPENDIX}}
    \medskip   


\section{Pure Untyped Lambda Calculus in Python}

Most untyped scripting languages support untyped lambda terms directly. We
summarize the creation of a basic programming language from pure lambda
calculus by implementing it using pure lambda terms in Python.
\begin{verbatim}
# Pure Lambda Calculus in Python (works in both Python2.2+ and Python3)

## The syntax of Python differs from traditional lambda calculus in that
# application is written f(x) instead of (f x), and a : is used for .

I = lambda x:x
K = lambda x:lambda y:x
S = lambda x:lambda y:lambda z:x(z)(y(z))

# Note: it would not be correct to define K as lambda x,y:x, because that's
# using python's built-in pairs.  Every lambda-abstraction takes exactly
# one argument, although that argument can be a tuple.  To be faithful to
# the build-from-scratch approach, we should use Curried functions and
# not rely on built-in features as much as possible: f(a,b) is not the same
# as f(a)(b).

true = K
false = K(I)
ifelse = lambda c:lambda a:lambda b:c(a)(b)
if_else = lambda c:lambda a:lambda b:c(a)(b)(I) # simulates call-by-name
NOT = lambda a:a(false)(true)

print(ifelse(NOT(true))(1)(2)) ### prints 2

# Church numerals 
ZERO = false
ONE = lambda f:lambda x:f(x)
PLUS = lambda m:lambda n:lambda f:lambda x:m(f)(n(f)(x))
TIMES = lambda m:lambda n:lambda f:lambda x:m(n(f))(x)

# linked lists
CONS = lambda a:lambda b:lambda c:c(a)(b)
CAR = lambda p:p(true)
CDR = lambda p:p(false)
NIL = ZERO
ISNIL = lambda p:p(lambda x:lambda y:lambda z:false)(true)

# applicative order FIX combinator for recursive definitions
FIX = lambda m:(lambda x:m(lambda y:x(x)(y)))(lambda x:(m(lambda y:x(x)(y))))

Primes = CONS(2)(CONS(3)(CONS(5)(CONS(7)(CONS(11)(NIL)))))
print(CAR(CDR(Primes)))  ### prints 3, the second prime

### functions to convert Python numerals to Church numerals:

def Church(n):  # converts non-neg integer n to Church numeral
    if (n==0): return ZERO
    elif n>0: return PLUS(ONE)(Church(n-1))
#
def Unchurch(c):  # converts Church numeral c to Python integer:
    return c(lambda y:1+y)(0)
#
## for example, Unchurch(TWO) beta-reduces to
# (lambda f:lambda x:f(f(x)))(lambda y:1+y)(0), which reduces to 1+1+0=2

# Unfortunately, Python cannot print intermediate lambda terms: it does
# does not support "higher order abstract syntax".

C25 = Church(25)  # pure lambda representation of 25
print(Unchurch(C25))  #### prints 25

## pure lambda function to add all numbers in a linked list:
SUM = FIX(lambda f:lambda n:if_else(ISNIL(n))(lambda y:ZERO)(
    lambda y:PLUS(CAR(n))(f(CDR(n)))))

L = CONS(Church(1))(CONS(Church(2))(CONS(Church(4))(CONS(Church(8))(NIL))))

ANSWER = SUM(L)
print(Unchurch(ANSWER))  ### prints 15
# end of program
\end{verbatim}




\end{document}
\medskip

The lambda calculus is the universal language of the programming language
research community. Hopefully this document has introduced you to a new way
to think about programming and programming languages.  


%%%%%%%%%%%%%%%
Now let's look at the term
$$ \lambda x.(\lambda y.(\lambda x.y+x)~3)~x+2$$
This example roughly corresponds to the first C program in the introduction.
There are two beta-redexes: the inner $(\lambda x.y+x)~3$ and the outer
$(\lambda y.(\lambda x.y+x)~3)~x+2$.  There are two correct ways to reduce
this term.  The first is to reduce the inner term first, which gives
$y+3$; then $\lambda x.(\lambda y.y+3)~x+2$ reduces to
$\lambda x.x+2+3$.  The second way is to first apply alpha conversion



\section{Static Single Assignment Form}

The lambda calculus is {\em stateless,\/} which means that the same
term will always reduce to the same normal form.  But what if we allowed
{\em destructive assignment\/} in our language: the infamous 
$x=x+1$?  Certainly computing this expression twice will have a
different effect than computing it once.  Such an opertion is not
stateless.  However, this does not necessarily mean that every program
that contains such assignments are not {\em admissible\/} in lambda
calculus.  In fact, many modern compilers will first translate a program
into an intermediate language called {\em Static Single
  Assignment Form\/} (SSA).  A new variable is introduced for each
destructive assignment, so $x=x+1$ is translated into something
like $x_2=x+1$. Loops containing such statements are translated
into {\em tail\/} recursive functions where $x$ is an argument.  The
popular {\em LLVM\/} intermediate language is based on SSA. 
The first C program on page 2, which contains a destructive assignment
$y=y+x$, is equivalent to the following pure lambda term:
$$
f= \lambda x. (\lambda y_2.y_2+x) ((\lambda y.(\lambda x.y+x)~2)~1)
$$
The destructive assignment to $y$ has been replaced by a {\em single\/}
binding to the variable $y_2$.  We see that $(f~4)$ reduces to
$(\lambda y_2.y_2+4) ((\lambda y.(\lambda x.y+x)~2)~1) ~\Rightarrow~$
$(\lambda y_2.y_2+4) ((\lambda y.y+2)~1) ~\Rightarrow~$
$(\lambda y_2.y_2+4) (1+2) ~\Longrightarrow~ 7$.
